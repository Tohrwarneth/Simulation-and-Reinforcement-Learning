%! suppress = PrimitiveEquation
%! suppress = FileNotFound
%------------------ vorlage.tex ------------------------------------------------
%
% LaTeX-Vorlage zur Erstellung von Projektdokumentationen
% im Fachbereich Informatik der Hochschule Trier
%
% Basis: Vorlage 'svmono' des Springer Verlags
% Bearbeiter: Hermann Schloß, Christian Bettinger
%
%-------------------------------------------------------------------------------


%------------------ Präambel ---------------------------------------------------
\documentclass[envcountsame, envcountchap, deutsch]{i-studis}

\usepackage[utf8]{inputenc}

\usepackage[a4paper]{geometry}
\usepackage[english, ngerman]{babel}

\usepackage[pdftex]{graphicx}
\usepackage{epstopdf}

\usepackage{listings}

\usepackage[german, ruled, vlined]{algorithm2e}
\usepackage{amssymb, amsfonts, amstext, amsmath}
\usepackage{array}
\usepackage[skip=10pt]{caption}
\usepackage[usenames, dvipsnames]{color}
\usepackage[pdftex, plainpages=false]{hyperref}
\usepackage{textcomp}

\usepackage{bibgerm}
\bibliographystyle{geralpha}

\usepackage{makeidx}
\usepackage{multicol}
\makeindex

\pagestyle{myheadings}
\setlength{\textheight}{1.1\textheight}

\lstset{
	basicstyle=\scriptsize\ttfamily,
	commentstyle=\scriptsize\ttfamily\color{Gray},
	identifierstyle=\scriptsize\ttfamily,
	keywordstyle=\scriptsize\ttfamily,
	stringstyle=\scriptsize\ttfamily,
	tabsize=4,
	numbers=left,
	numberstyle=\tiny,
	numberblanklines=false,
	frame=single,
	framesep=3mm,
	framexleftmargin=7mm,
	xleftmargin=10mm,
	linewidth=144mm,
	captionpos=b,
}

\usepackage[Export]{adjustbox}
\adjustboxset*{width=1\linewidth}


\hypersetup{
$if(title-meta)$
  pdftitle={$title-meta$},
$endif$
$if(author-meta)$
  pdfauthor={$author-meta$},
$endif$
$if(lang)$
  pdflang={$lang$},
$endif$
$if(subject)$
  pdfsubject={$subject$},
$endif$
$if(keywords)$
  pdfkeywords={$for(keywords)$$keywords$$sep$, $endfor$},
$endif$
}

\providecommand{\tightlist}{%
  \setlength{\itemsep}{0pt}\setlength{\parskip}{0pt}}

%------------------ Manuelle Silbentrennung ------------------------------------
\usepackage[T1]{fontenc}
\usepackage{amsmath}
\hyphenation{Ele-men-tar-ob-jek-te ab-ge-tas-tet Aus-wer-tung House-holder-Matrix Least-Squares-Al-go-ri-th-men grenz-über-schrei-tende}


%------------------ Titelseite -------------------------------------------------
\begin{document}

$if(status)$
$if(title)$
\title{{\color{red} \huge \textit{$status$}\newline}$title$}
$endif$
$else$
$if(title)$
\title{$title$}
$endif$
$endif$
$if(subtitle)$
\subtitle{$subtitle$}
$endif$

$if(author)$
\author{$author$}
$endif$

$if(supervisor)$
\supervisor{$supervisor$}
$endif$

$if(address)$
\address{$address$}
$endif$
$if(date)$
\submitdate{$date$}
$endif$

%------------------ Projektart -------------------------------------------------
%\project{Bachelor-Projektarbeit}
% \project{Bachelor-Abschlussarbeit}
% \project{Master-Projektstudium}
%\project{Master-Abschlussarbeit}
%\project{Seminar}
%\project{Hausarbeit}
$if(project)$
\project{$project$}
$endif$

\mytitlepage

%------------------ Vorwort, Kurzfassung, Verzeichnisse ------------------------
\frontmatter
% \input{chapters/Vorwort}								% Vorwort (optional)
% \input{chapters/Kurzfassung}							% Kurzfassung/Abstract
$if(abstract)$
\abstract{$abstract$}
$endif$
\tableofcontents										% Inhaltsverzeichnis
% \listoffigures											% Abbildungsverzeichnis (optional)
% \listoftables											% Tabellenverzeichnis (optional)
% \lstlistoflistings										% Listings (optional)


%------------------ Kapitel ----------------------------------------------------
\mainmatter
$body$


% %------------------ Literaturverzeichnis & Index -------------------------------
\backmatter
$if(bib)$
\bibliography{$bib$}								% Literaturverzeichnis (literatur.bib)
$endif$

\printindex												% Index (optional)


% %------------------ Anhänge ----------------------------------------------------
% \begin{appendix}
% 	\include{chapters/Glossar}							% Glossar (optional)
% 	\include{chapters/Selbststaendigkeitserklaerung}	% Selbstständigkeitserklärung
% \end{appendix}


\end{document}
