%------------------ vorlage.tex ------------------------------------------------
%
% LaTeX-Vorlage zur Erstellung von Projektdokumentationen
% im Fachbereich Informatik der Hochschule Trier
%
% Basis: Vorlage 'svmono' des Springer Verlags
% Bearbeiter: Hermann Schloß, Christian Bettinger
%
%-------------------------------------------------------------------------------


%------------------ Präambel ---------------------------------------------------
\documentclass[envcountsame, envcountchap, deutsch]{i-studis}

\usepackage[utf8]{inputenc}

\usepackage[a4paper]{geometry}
\usepackage[english, ngerman]{babel}

\usepackage[pdftex]{graphicx}
\usepackage{epstopdf}

\usepackage{listings}

\usepackage[german, ruled, vlined]{algorithm2e}
\usepackage{amssymb, amsfonts, amstext, amsmath}
\usepackage{array}
\usepackage[skip=10pt]{caption}
\usepackage[usenames, dvipsnames]{color}
\usepackage[pdftex, plainpages=false]{hyperref}
\usepackage{textcomp}

\usepackage{bibgerm}
\bibliographystyle{geralpha}

\usepackage{makeidx}
\usepackage{multicol}
\makeindex

\pagestyle{myheadings}
\setlength{\textheight}{1.1\textheight}

\lstset{
	basicstyle=\scriptsize\ttfamily,
	commentstyle=\scriptsize\ttfamily\color{Gray},
	identifierstyle=\scriptsize\ttfamily,
	keywordstyle=\scriptsize\ttfamily,
	stringstyle=\scriptsize\ttfamily,
	tabsize=4,
	numbers=left,
	numberstyle=\tiny,
	numberblanklines=false,
	frame=single,
	framesep=3mm,
	framexleftmargin=7mm,
	xleftmargin=10mm,
	linewidth=144mm,
	captionpos=b,
}

\usepackage[Export]{adjustbox}
\adjustboxset*{width=1\linewidth}


\hypersetup{
  pdftitle={Reinforcement Learning},
  pdfauthor={Frederic Nicolas Schneider},
  pdflang={de-DE},
  pdfsubject={Simulationstechnik und Reinforcement Learning},
}

\providecommand{\tightlist}{%
  \setlength{\itemsep}{0pt}\setlength{\parskip}{0pt}}

%------------------ Manuelle Silbentrennung ------------------------------------
\hyphenation{Ele-men-tar-ob-jek-te ab-ge-tas-tet Aus-wer-tung House-holder-Matrix Least-Squares-Al-go-ri-th-men}


%------------------ Titelseite -------------------------------------------------
\begin{document}

\title{Reinforcement Learning}
\subtitle{Simulationstechnik und Reinforcement Learning}

\author{Frederic Nicolas Schneider}

\supervisor{Prof.~Dr.~Christoph Lürig}

\address{Trier}
\submitdate{11.07.2023}

%------------------ Projektart -------------------------------------------------
%\project{Bachelor-Projektarbeit}
% \project{Bachelor-Abschlussarbeit}
% \project{Master-Projektstudium}
%\project{Master-Abschlussarbeit}
%\project{Seminar}
%\project{Hausarbeit}
\project{Simulationstechnik und Reinforcement Learning}

\mytitlepage

%------------------ Vorwort, Kurzfassung, Verzeichnisse ------------------------
\frontmatter
% \input{chapters/Vorwort}								% Vorwort (optional)
% \input{chapters/Kurzfassung}							% Kurzfassung/Abstract
\tableofcontents										% Inhaltsverzeichnis
% \listoffigures											% Abbildungsverzeichnis (optional)
% \listoftables											% Tabellenverzeichnis (optional)
% \lstlistoflistings										% Listings (optional)


%------------------ Kapitel ----------------------------------------------------
\mainmatter
\newpage

\hypertarget{ruxfcckblick-auf-die-simulation}{%
\chapter{Rückblick auf die
Simulation}\label{ruxfcckblick-auf-die-simulation}}

Die Simulation wurde in drei Elementen aufgeteilt: der
Fahrstuhlsteuerung, Personensteuerung und der verbindenden Simulation.
Dabei wird in jedem Takt überprüft, ob Personen einen Fahrstuhl zu einer
Etage rufen. Ist der Fahrstuhl leer, prüft er in jedem Takt, ob er
gerufen wurde. Hat er jedoch Passagiere, so fährt er zur nächsten
Zieletage und nimmt, wenn möglich, auf dem Weg weitere Passagiere mit.


% %------------------ Literaturverzeichnis & Index -------------------------------
% \backmatter
% \bibliography{literatur}								% Literaturverzeichnis (literatur.bib)
% \printindex												% Index (optional)


% %------------------ Anhänge ----------------------------------------------------
% \begin{appendix}
% 	\include{chapters/Glossar}							% Glossar (optional)
% 	\include{chapters/Selbststaendigkeitserklaerung}	% Selbstständigkeitserklärung
% \end{appendix}


\end{document}
